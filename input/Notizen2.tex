\section{Notizen}

\subsection{09.01.}

Verbindung mit RFU6
\begin{itemize}
    \item Sopas
    \begin{itemize}
        \item [] Geräte suche nach dem man Ethernet 9 auf 192.168.136.1 gestellt
        \item [] IP-Adresse auf 192.168.136.2 ändern
        \item [] Geräte-Fenster öffnen und auf Service umstellen mit dem Passwort:servicelevel
    \end{itemize}
    \item CTT
    \begin{itemize}
        \item [] Öffne das .ctt.xml File
        \item [] Gehe auf Projekt dann auf Settings und dann auf Server test
        \item [] Hier wird die Ip-Adresse auf die des Sensors gesetzt
        \item [] Am besten ist der Sensor mit keinem anderen Geräte verbunden
    \end{itemize}
\end{itemize}

\subsection{10.01.}
Dokumentation zu OPC \url{https://www.open62541.org/doc/master/toc.html}

\subsection{11.01}
RFU6 Open625441Client läuft auf Viual Studio alles unnötige wurde gelöscht. 
Muss wahrscheinlich neu geöffnet werden. Eigenschaft einfach aus dem alten Projekt kopieren.\\
xml einlesen ca. 30 min \\
Geschriebenes xml kann von Maschienen und von Menschen gelesen werden und wird dazu verwendet Typen zu definieren. 
Es kann online formatiert werden. Zudem kann man sein xml validieren in dem man es in ein Schema online umwandelt.\\
\lstinputlisting[language=XML]{input/simplexml.xml}
\subsubsection*{node.js}
\begin{itemize}
    \item Laufzeitumgebung für JavaScript
    \item keine Programmiersprache
    \item Backend-Services(API)
\end{itemize}

\subsection{12.01}
JavaScript dann Typescript und dann node.js 
\subsubsection*{Framework für JavaScript}
Vue.js \\
Sehr ähnlich zu C und C++. Wie funktioniert die Datenstruktur JSON? 


\subsection{13.01}
Klassen können aus anderen Datein imporiert werden\\ import \{ Mensch \} from "./Mensch.js";

Evaluierung einer bestehenden OPC UA Server Implementierung zur mögliuchen erreichung der Konformität gemäß \href{https://opcfoundation.org/developer-tools/certification-test-tools/opc-ua-compliance-test-tool-uactt/}{https://opcfoundation.org}
